\documentclass[11pt]{article}
\usepackage[margin=1in]{geometry}
\usepackage{amsmath,amssymb,amsthm}
\usepackage{graphicx}
\usepackage{booktabs}
\usepackage{natbib}
\usepackage{hyperref}
\usepackage{float}

\title{MLB Salary Efficiency Analysis: \\A Bayesian Hierarchical Approach}
\author{Andrew Bai\\University of Chicago}
\date{\today}

\begin{document}

\maketitle

\begin{abstract}
We analyze MLB salary efficiency using real contract data from the Lahman database (2000--2016) merged with FanGraphs WAR. Key findings from 8,198 player-seasons: the market rate rose from \$1.3M/WAR (2000) to \$2.8M/WAR (2016); first basemen are most overpaid (\$2.6M/WAR, 41\% above average) while second basemen are cheapest (\$1.4M/WAR); star players ($>$5 WAR) are actually paid \emph{less} per WAR than average players due to service-time rules; market-clearing teams (Oakland, Miami, Tampa Bay) pay 40--60\% less per WAR than large-market clubs; and contrary to the ``Moneyball'' narrative, market efficiency has \emph{not} improved---R$^2$ of salary-WAR regressions actually declined. However, a decision-theoretic analysis reveals that apparent ``inefficiency'' is partly rational: the correlation between implied and rational win values is 0.50 ($p = 0.005$), and rationality scores have improved from 0.31 (2000) to 0.49 (2016). Teams increasingly allocate spending based on market size and playoff contention, even if individual contracts don't perfectly track WAR. The market is neither perfectly efficient nor irrational---it reflects heterogeneous team objectives and institutional constraints.
\end{abstract}

\section{Introduction}

Professional sports provide a natural laboratory for studying labor market efficiency. Unlike most markets, player productivity in baseball is precisely measured through statistics like Wins Above Replacement (WAR), and salaries are publicly observable. This paper asks four questions:

\begin{enumerate}
    \item What is the market rate for wins, and how has it evolved?
    \item Are certain positions or player types systematically mispriced?
    \item Which teams achieve better value in the salary market?
    \item Has market efficiency improved over time (``Does management learn?'')
\end{enumerate}



\subsection{Limitations}

Several caveats apply:
\begin{itemize}
    \item Lahman salary data ends in 2016; recent trends are not captured.
    \item We observe annual salary, not contract terms (years, guarantees, deferrals).
    \item ``Efficiency'' means salary per WAR, not team optimality---large-market teams may rationally pay premiums.
    \item WAR embeds modeling assumptions about replacement level.
\end{itemize}

\section{Data}

\subsection{Sources}

We merge two datasets:
\begin{itemize}
    \item \textbf{Lahman Salaries}: Real observed player salaries, 2000--2016
    \item \textbf{FanGraphs WAR}: Player productivity metrics via \texttt{pybaseball}
\end{itemize}

Matching on player name and team-year yields 8,198 player-seasons with WAR $\geq 0.1$.

\subsection{Sample Statistics}

\begin{table}[H]
\centering
\caption{Sample Statistics (Real Salary Data)}
\label{tab:sample}
\begin{tabular}{lr}
\toprule
Metric & Value \\
\midrule
Total player-seasons & 8,198 \\
Unique teams & 30 \\
Seasons & 2000--2016 \\
\midrule
Salary distribution: & \\
\quad Min & \$165,574 \\
\quad Median & \$1,600,000 \\
\quad Mean & \$3,724,070 \\
\quad Max & \$33,000,000 \\
\midrule
WAR distribution: & \\
\quad Mean & 1.96 \\
\quad Median & 1.40 \\
\quad Range & [0.1, 12.7] \\
\bottomrule
\end{tabular}
\end{table}

\section{Descriptive Analysis}

\subsection{Market Rate for Wins}

The market rate (total salary / total WAR) has more than doubled over our sample period:

\begin{table}[H]
\centering
\caption{Market Rate for Wins Over Time}
\label{tab:marketrate}
\begin{tabular}{lrrr}
\toprule
Year & \$/WAR & Avg Salary & N \\
\midrule
2000 & \$1.26M & \$2.47M & 488 \\
2004 & \$1.54M & \$3.11M & 481 \\
2008 & \$1.88M & \$3.67M & 480 \\
2012 & \$2.15M & \$4.02M & 489 \\
2016 & \$2.76M & \$5.11M & 499 \\
\bottomrule
\end{tabular}
\end{table}

The \$/WAR grew 118\% from 2000 to 2016, reflecting both revenue growth and competitive bidding for talent.

\subsection{Position Pricing}

\begin{table}[H]
\centering
\caption{Position Pricing (Dollars per WAR)}
\label{tab:position}
\begin{tabular}{lrrrr}
\toprule
Position & \$/WAR & Avg Salary & N & Deviation \\
\midrule
1B & \$2.57M & \$6.07M & 545 & +41\% \\
P & \$2.00M & \$3.32M & 3,918 & +9\% \\
OF & \$1.89M & \$4.35M & 1,597 & +3\% \\
3B & \$1.72M & \$4.26M & 519 & $-$6\% \\
SS & \$1.66M & \$3.66M & 485 & $-$9\% \\
C & \$1.52M & \$2.70M & 619 & $-$17\% \\
2B & \$1.43M & \$3.11M & 515 & $-$22\% \\
\bottomrule
\end{tabular}
\end{table}

\textbf{Key findings by position}:
\begin{itemize}
    \item \textbf{First basemen are most expensive}: \$2.57M/WAR, 41\% above average. This may reflect: (a) offensive production is more visible/marketable, (b) selection effects (1B WAR requires elite hitting to overcome negative positional adjustment).

    \item \textbf{Pitchers command premium}: \$2.00M/WAR, 9\% above average. Reflects injury risk and volatility.

    \item \textbf{Middle infielders are cheapest}: 2B (\$1.43M) and SS (\$1.66M) are 20\%+ below average, despite SS being a premium defensive position. This suggests service-time effects---many elite SS are young and cost-controlled.

    \item \textbf{Catchers undervalued}: \$1.52M/WAR despite the position's scarcity. May reflect shorter careers and injury risk discounting.
\end{itemize}

\subsection{Star Premium (or Lack Thereof)}

Conventional wisdom holds that star players command a premium per win due to scarcity. Our data tell a different story:

\begin{table}[H]
\centering
\caption{Star Premium: \$/WAR by Performance Tier}
\label{tab:star}
\begin{tabular}{lrrr}
\toprule
WAR Bucket & \$/WAR & Avg Salary & N \\
\midrule
0--1 WAR & \$4.67M & \$2.39M & 3,245 \\
1--2 WAR & \$2.21M & \$3.33M & 1,901 \\
2--3 WAR & \$1.78M & \$4.47M & 1,209 \\
3--4 WAR & \$1.46M & \$5.09M & 815 \\
4--5 WAR & \$1.33M & \$6.01M & 488 \\
5+ WAR & \$1.15M & \$7.34M & 540 \\
\bottomrule
\end{tabular}
\end{table}

\textbf{Key finding}: The \$/WAR \emph{decreases} monotonically with performance. Stars (5+ WAR) are paid only \$1.15M per WAR compared to \$4.67M for marginal players (0--1 WAR). This is a 4:1 ratio in the \emph{opposite} direction of conventional wisdom.

\textbf{Explanation}: This reflects \emph{selection effects}. Low-WAR players in our sample are overpaid veterans on guaranteed contracts whose performance declined, pre-arbitration minimums for players who barely contributed, or injury cases. High-WAR players include cost-controlled pre-arb and arbitration players (like Mike Trout earning \$500K while producing 9 WAR).

\subsection{Does Management Learn? Market Efficiency Over Time}

We test whether the salary market has become more efficient by examining how well salary predicts WAR (higher R$^2$ = more efficient pricing):

\begin{table}[H]
\centering
\caption{Market Efficiency by Era}
\label{tab:efficiency}
\begin{tabular}{lrrr}
\toprule
Era & R$^2$ & Residual Var & N \\
\midrule
2000--2004 & 0.127 & 1.53 & 2,401 \\
2005--2008 & 0.065 & 1.62 & 1,924 \\
2009--2012 & 0.076 & 1.53 & 1,901 \\
2013--2016 & 0.073 & 1.58 & 1,972 \\
\bottomrule
\end{tabular}
\end{table}

\textbf{Key finding}: Market efficiency has \emph{not} improved. The R$^2$ of salary-WAR regressions actually \emph{declined} from 0.127 in 2000--2004 to 0.073 in 2013--2016.

Formal trend test:
\begin{itemize}
    \item R$^2$ trend: $-0.0038$/year ($p = 0.008$)
    \item RMSE trend: $+0.0008$/year ($p = 0.61$)
\end{itemize}

The negative R$^2$ trend is statistically significant, contradicting the ``Moneyball'' narrative that analytics have made the market more efficient. Possible explanations:
\begin{itemize}
    \item Guaranteed contracts lock in salaries regardless of performance
    \item Arbitration/pre-arb system decouples salary from current productivity
    \item Teams may value projected future performance, not current WAR
    \item Risk preferences and injury history affect pricing
\end{itemize}

\section{Bayesian Hierarchical Model}

We develop a fully Bayesian hierarchical model that addresses several limitations of naive salary-WAR regressions: measurement error in WAR, smooth market inflation, nonlinear star premiums, and heteroskedastic residuals.

\subsection{Model Specification}

Let $i = 1, \ldots, N$ index player-seasons, with team $t_i \in \{1, \ldots, 30\}$ and year $y_i \in \{2000, \ldots, 2016\}$.

\subsubsection{Latent Talent (Measurement Error Model)}

WAR is a noisy estimate of true player value. We model observed WAR as a draw from a distribution centered at latent talent $\theta_i$:
\begin{equation}
    \text{WAR}_i \mid \theta_i \sim N(\theta_i, \sigma_{\text{WAR}}^2)
\end{equation}
where $\sigma_{\text{WAR}} = 0.5$ reflects the typical standard error of WAR estimates. This measurement error model shrinks extreme WAR observations toward the mean and provides appropriate uncertainty quantification.

\subsubsection{Salary Model}

Log salary follows a linear model in latent talent with piecewise slope:
\begin{equation}
    \log(\text{salary}_i) = \alpha + \beta_1 \theta_i + \beta_2 (\theta_i - \kappa)_+ + \gamma_{\text{team}}[t_i] + \gamma_{\text{year}}[y_i] + \epsilon_i
\end{equation}
where:
\begin{itemize}
    \item $\alpha$ is the baseline log salary intercept
    \item $\beta_1$ is the base salary-WAR elasticity (for $\theta_i < \kappa$)
    \item $\beta_2$ is the additional slope for stars (for $\theta_i \geq \kappa$), so total slope above the knot is $\beta_1 + \beta_2$
    \item $\kappa = 3$ WAR is the ``star threshold'' where pricing may change
    \item $(x)_+ = \max(x, 0)$ is the positive part function
    \item $\gamma_{\text{team}}[t]$ is the team-specific effect (positive = overpays)
    \item $\gamma_{\text{year}}[y]$ is the year effect (market inflation)
\end{itemize}

\subsubsection{Team Effects (Partial Pooling)}

Team effects are drawn from a common distribution, providing regularization:
\begin{equation}
    \gamma_{\text{team}}[t] \mid \sigma_{\text{team}} \stackrel{iid}{\sim} N(0, \sigma_{\text{team}}^2), \quad t = 1, \ldots, 30
\end{equation}
This hierarchical structure shrinks extreme team estimates toward zero, appropriate given the modest sample sizes per team.

\subsubsection{Year Effects (Random Walk)}

Rather than independent year effects, we impose temporal smoothness via a random walk prior:
\begin{align}
    \gamma_{\text{year}}[1] &\sim N(0, 0.5^2) \\
    \gamma_{\text{year}}[t] \mid \gamma_{\text{year}}[t-1] &\sim N(\gamma_{\text{year}}[t-1], \tau^2), \quad t = 2, \ldots, T
\end{align}
This ensures market inflation evolves smoothly year-to-year, rather than jumping erratically.

\subsubsection{Heteroskedastic Residuals}

Salary variance plausibly scales with player value. We model:
\begin{equation}
    \epsilon_i \mid \theta_i \sim N(0, \sigma_i^2), \quad \log \sigma_i = a + b \cdot \theta_i
\end{equation}
This allows residual variance to increase (or decrease) with latent talent.

\subsection{Prior Distributions}

We specify weakly informative priors:
\begin{align}
    \alpha &\sim N(14, 2^2) && \text{(baseline $\approx$ \$1.2M)} \\
    \beta_1 &\sim N(0.3, 0.2^2) && \text{(exp(0.3) $\approx$ 1.35$\times$ per WAR)} \\
    \beta_2 &\sim N(0.1, 0.1^2) && \text{(modest star premium expected)} \\
    \sigma_{\text{team}} &\sim \text{Half-}N(0, 0.2^2) && \text{(team effect scale)} \\
    \tau &\sim \text{Half-}N(0, 0.1^2) && \text{(year RW innovation)} \\
    a &\sim N(-0.5, 0.5^2) && \text{(baseline residual SD)} \\
    b &\sim N(0, 0.1^2) && \text{(heteroskedasticity slope)}
\end{align}

The joint posterior is:
\begin{equation}
    p(\boldsymbol{\theta}, \boldsymbol{\gamma}, \boldsymbol{\phi} \mid \mathbf{y}, \mathbf{WAR}) \propto \prod_i p(\text{WAR}_i \mid \theta_i) \cdot p(y_i \mid \theta_i, \boldsymbol{\gamma}, \boldsymbol{\phi}) \cdot p(\boldsymbol{\gamma}) \cdot p(\boldsymbol{\phi})
\end{equation}
where $\boldsymbol{\phi} = (\alpha, \beta_1, \beta_2, a, b, \sigma_{\text{team}}, \tau)$ collects the global parameters.

\subsection{Posterior Computation}

We sample from the posterior using a Metropolis-within-Gibbs algorithm:
\begin{enumerate}
    \item \textbf{Scalar parameters} $(\alpha, \beta_1, \beta_2, a, b)$: Metropolis updates with Gaussian proposals
    \item \textbf{Variance parameters} $(\sigma_{\text{team}}, \tau)$: Metropolis on log scale to enforce positivity
    \item \textbf{Latent talent} $\boldsymbol{\theta}$: Block Metropolis updates (blocks of 100 observations)
    \item \textbf{Team effects} $\boldsymbol{\gamma}_{\text{team}}$: Component-wise Metropolis
    \item \textbf{Year effects} $\boldsymbol{\gamma}_{\text{year}}$: Component-wise Metropolis
\end{enumerate}

We run 4,000 iterations (1,000 warmup + 3,000 retained). Proposal standard deviations are hand-tuned to achieve $\sim$25\% acceptance rates.

\subsection{Interpretation of Parameters}

\textbf{Team efficiency} is defined as $\exp(-\gamma_{\text{team}}[t])$:
\begin{itemize}
    \item If $\gamma_{\text{team}} < 0$: team pays less than average for equivalent talent (efficient)
    \item If $\gamma_{\text{team}} > 0$: team pays more than average (inefficient)
    \item Efficiency $> 1$ means the team gets more WAR per dollar
\end{itemize}

\textbf{Star premium}: If $\beta_2 > 0$, stars command higher $/WAR; if $\beta_2 < 0$, stars are bargains.

\textbf{Year effects}: $\gamma_{\text{year}}[2016] - \gamma_{\text{year}}[2000]$ gives total market inflation after controlling for WAR and teams.

\section{Model Results}

\subsection{Fixed Effects}

\begin{table}[H]
\centering
\caption{Posterior Estimates for Fixed Effects}
\label{tab:fixed}
\begin{tabular}{lrrr}
\toprule
Parameter & Post.\ Mean & Post.\ SD & Interpretation \\
\midrule
$\alpha$ (intercept) & 13.73 & 0.14 & Baseline log salary \\
$\beta_1$ (base WAR) & 0.28 & 0.02 & 1.32$\times$ per WAR (below 3) \\
$\beta_2$ (star effect) & $-$0.12 & 0.03 & 1.17$\times$ per WAR (above 3) \\
\bottomrule
\end{tabular}
\end{table}

The negative $\beta_2$ confirms the descriptive finding: salary growth per WAR \emph{slows} for elite players. Each additional WAR multiplies salary by 1.32$\times$ for players below 3 WAR, but only 1.17$\times$ above that threshold.

\subsection{Team Efficiency Rankings}

Team efficiency = $\exp(-\gamma_{\text{team}})$: teams with negative $\gamma$ pay less for equivalent talent.

\begin{table}[H]
\centering
\caption{Team Efficiency Rankings (Top/Bottom 5)}
\label{tab:teams}
\begin{tabular}{lrrrr}
\toprule
Team & $\gamma$ Mean & Efficiency & 95\% CI & P(Above Avg) \\
\midrule
\multicolumn{5}{l}{\textit{Most Efficient (pay less per WAR):}} \\
MIA & $-$0.45 & 1.58 & [1.33, 1.88] & 100\% \\
OAK & $-$0.41 & 1.51 & [1.26, 1.80] & 100\% \\
PIT & $-$0.37 & 1.45 & [1.21, 1.73] & 100\% \\
TBR & $-$0.36 & 1.44 & [1.19, 1.72] & 100\% \\
SDP & $-$0.27 & 1.31 & [1.12, 1.57] & 99.8\% \\
\midrule
\multicolumn{5}{l}{\textit{Least Efficient (pay more per WAR):}} \\
CHC & 0.29 & 0.75 & [0.63, 0.88] & 0.0\% \\
NYM & 0.31 & 0.74 & [0.62, 0.87] & 0.0\% \\
LAD & 0.38 & 0.68 & [0.57, 0.82] & 0.0\% \\
BOS & 0.46 & 0.63 & [0.53, 0.75] & 0.0\% \\
NYY & 0.75 & 0.48 & [0.40, 0.56] & 0.0\% \\
\bottomrule
\end{tabular}
\end{table}

The efficiency gap is substantial: Oakland obtains roughly 3$\times$ more WAR per dollar than the Yankees. This reflects both payroll constraints and organizational philosophy.

\textbf{Caveat}: ``Inefficiency'' may be rational for large-market teams if their marginal win value is higher (playoff odds, revenue, brand value). The Yankees paying premium prices may still be optimal given their revenue structure.

\subsection{Year Effects (Market Inflation)}

\begin{table}[H]
\centering
\caption{Year Effects (Random Walk)}
\label{tab:years}
\begin{tabular}{lrr}
\toprule
Year & $\gamma_{\text{year}}$ Mean & 95\% CI \\
\midrule
2000 & $-$0.23 & [$-$0.43, 0.10] \\
2004 & $-$0.07 & [$-$0.27, 0.26] \\
2008 & 0.07 & [$-$0.12, 0.41] \\
2012 & 0.22 & [0.02, 0.56] \\
2016 & 0.40 & [0.20, 0.73] \\
\bottomrule
\end{tabular}
\end{table}

Controlling for WAR and team, salaries increased by $\exp(0.63) \approx 1.9\times$ from 2000 to 2016, or about 4\% annually.

\section{Decision-Theoretic Analysis: Is ``Inefficiency'' Optimal?}

The efficiency framing asks whether teams pay fair value for WAR. But this ignores that teams face heterogeneous optimization problems. A large-market team near playoff contention may rationally pay more per win than a rebuilding small-market team.

\subsection{Theoretical Framework}

Each team maximizes expected utility:
\begin{equation}
    \max_{\text{roster}} \mathbb{E}[U_t(\text{wins})] - C(\text{roster})
\end{equation}
where $U_t(\cdot)$ is team-specific utility capturing:
\begin{itemize}
    \item \textbf{Regular season revenue}: Larger markets have higher revenue per win
    \item \textbf{Playoff option value}: Making the playoffs yields additional revenue and brand value
    \item \textbf{Championship probability}: Deep playoff runs are highly valuable
\end{itemize}

The \textbf{marginal win value} for team $t$ is:
\begin{equation}
    v_t = \underbrace{\frac{\partial R_t}{\partial W}}_{\text{revenue effect}} + \underbrace{B \cdot \frac{\partial P_t(\text{playoffs})}{\partial W}}_{\text{playoff option}}
\end{equation}
where $R_t$ is regular-season revenue, $P_t(\text{playoffs})$ is playoff probability, and $B$ is the bonus from making playoffs.

\subsection{Playoff Probability Model}

We estimate playoff probability as a function of wins using logistic regression on 2000--2016 data:
\begin{equation}
    P(\text{playoffs} \mid W) = \frac{1}{1 + \exp(-(\alpha + \beta W))}
\end{equation}

\begin{table}[H]
\centering
\caption{Playoff Probability by Win Total}
\label{tab:playoff}
\begin{tabular}{lr}
\toprule
Wins & P(Playoffs) \\
\midrule
81 & 1.3\% \\
85 & 11.6\% \\
90 & 65.3\% \\
95 & 96.8\% \\
\bottomrule
\end{tabular}
\end{table}

The marginal playoff probability $dP/dW$ peaks around 88--90 wins, where an extra win has the largest impact on playoff odds.

\subsection{Market Size and Rational Win Values}

We use average attendance as a proxy for market size. Combined with the playoff probability model, we compute each team's \textbf{rational marginal win value}:

\begin{table}[H]
\centering
\caption{Market Size and Rational Win Values (Selected Teams)}
\label{tab:rational}
\begin{tabular}{lrrr}
\toprule
Team & Market Index & Avg Wins & Rational Win Index \\
\midrule
\multicolumn{4}{l}{\textit{Large Markets (high rational value):}} \\
BOS & 1.22 & 91.5 & 2.52 \\
LAA & 1.23 & 87.5 & 2.37 \\
LAD & 1.42 & 86.4 & 2.28 \\
NYY & 1.48 & 94.2 & 1.38 \\
\midrule
\multicolumn{4}{l}{\textit{Small Markets (low rational value):}} \\
TBR & 0.58 & 80.9 & 0.58 \\
MIA & 0.61 & 73.8 & 0.60 \\
OAK & 0.75 & 84.5 & 0.74 \\
\bottomrule
\end{tabular}
\end{table}

Note that NYY has a lower rational value than other large markets because their high average wins (94.2) put them past the steep part of the playoff curve---they're already likely to make playoffs, so marginal wins matter less.

\subsection{Testing Rationality: Implied vs.\ Rational Win Values}

We compare teams' \textbf{implied} win values (backed out from $\gamma_{\text{team}}$) to their \textbf{rational} values:
\begin{equation}
    v_t^{\text{implied}} = \bar{v} \cdot \exp(\gamma_{\text{team}}[t])
\end{equation}

If spending patterns reflect rational optimization, implied and rational values should correlate positively.

\begin{table}[H]
\centering
\caption{Rationality Test Results}
\label{tab:rationality}
\begin{tabular}{lr}
\toprule
Statistic & Value \\
\midrule
Correlation(implied, rational) & 0.50 \\
p-value & 0.005 \\
$R^2$ & 0.25 \\
\bottomrule
\end{tabular}
\end{table}

The correlation of 0.50 is statistically significant, suggesting that roughly 25\% of cross-team spending variation is explained by rational factors.

\subsection{Irrational Overpayers vs.\ Rational Underspenders}

Some teams' spending patterns deviate from rational predictions:

\textbf{Irrational overpayers} (high implied value, low rational value):
\begin{itemize}
    \item NYY: Implied 2.03$\times$, rational 1.38$\times$ (gap: +0.65)
    \item NYM: Implied 1.31$\times$, rational 0.80$\times$ (gap: +0.51)
    \item CHC: Implied 1.29$\times$, rational 0.82$\times$ (gap: +0.47)
\end{itemize}

\textbf{Rational underspenders} (low implied value, high rational value---leaving value on table):
\begin{itemize}
    \item LAA: Implied 1.02$\times$, rational 2.37$\times$ (gap: $-$1.36)
    \item BOS: Implied 1.52$\times$, rational 2.52$\times$ (gap: $-$1.00)
    \item STL: Implied 1.07$\times$, rational 2.07$\times$ (gap: $-$0.99)
\end{itemize}

The Angels, Red Sox, and Cardinals appear to underspend relative to their market position and playoff potential.

\subsection{Does Rationality Improve Over Time?}

While market ``efficiency'' (R$^2$ of salary-WAR) declined over our sample period, we can separately ask whether team \emph{decisions} have become more rational---i.e., more aligned with market size and playoff contention.

For each year, we compute a ``rationality score'': the average correlation between team spending and rational factors (attendance and wins). Higher scores indicate spending patterns more aligned with rational optimization.

\begin{table}[H]
\centering
\caption{Rationality Trend Over Time}
\label{tab:rationality_trend}
\begin{tabular}{lrr}
\toprule
Year & Rationality Score & Corr(Spending, Wins) \\
\midrule
2000 & 0.31 & 0.21 \\
2004 & 0.36 & 0.28 \\
2008 & 0.42 & 0.35 \\
2012 & 0.45 & 0.38 \\
2016 & 0.49 & 0.42 \\
\bottomrule
\end{tabular}
\end{table}

\textbf{Key finding}: Rationality scores increased from 0.31 (2000) to 0.49 (2016), a 58\% improvement. The trend is positive (+0.007/year) but not statistically significant at conventional levels ($p = 0.21$).

This suggests that while overall ``efficiency'' (salary predicting WAR) has not improved, \emph{strategic decision-making} may be improving. Teams are increasingly allocating spending based on market size and competitive position, even if individual player contracts don't perfectly track WAR.

The distinction is important: efficiency measures whether teams pay fair value for talent, while rationality measures whether spending allocation matches team objectives. A team can be ``inefficient'' (overpaying for WAR) but ``rational'' (overpaying because their marginal win value is high).

\subsection{Implications}

\begin{enumerate}
    \item \textbf{``Inefficiency'' is partly rational}: Large-market teams paying premiums may be optimizing correctly given their revenue structure.

    \item \textbf{But not entirely}: 75\% of spending variance remains unexplained by our rational model, suggesting behavioral factors, agency problems, or unmeasured variables.

    \item \textbf{Small-market ``efficiency'' may be suboptimal}: Teams like Oakland and Tampa Bay achieve high WAR per dollar but may be leaving playoff probability on the table.

    \item \textbf{The market is not irrational}: The significant correlation suggests teams respond to incentives, even if imperfectly.

    \item \textbf{Decisions are improving}: While not statistically significant, the positive trend in rationality scores suggests teams are getting better at strategic resource allocation.
\end{enumerate}

\section{Discussion}

\subsection{Summary of Findings}

\begin{enumerate}
    \item \textbf{Market rate doubled}: \$/WAR grew from \$1.3M (2000) to \$2.8M (2016).

    \item \textbf{Pitchers command premium}: Pitchers are paid 15\% more per WAR than position players, likely reflecting injury risk and volatility.

    \item \textbf{No star premium}: Contrary to conventional wisdom, stars are paid \emph{less} per WAR than average players, due to service-time rules and selection effects.

    \item \textbf{Large team effects}: Market-clearing teams (OAK, TBR, MIA) pay 40--60\% less per WAR than large-market clubs (NYY, BOS, LAD).

    \item \textbf{No efficiency improvement}: Market efficiency has not improved over time; R$^2$ of salary-WAR regressions actually declined.
\end{enumerate}

\subsection{Why Hasn't the Market Become More Efficient?}

The ``Moneyball'' narrative suggests that as analytics spread, inefficiencies should be arbitraged away. Our data show the opposite. Possible explanations:

\begin{enumerate}
    \item \textbf{Institutional constraints}: The service-time system (pre-arb, arbitration, free agency) prevents salaries from clearing at market rates. Cost-controlled players produce surplus value regardless of efficiency.

    \item \textbf{Contract structure}: Multi-year guaranteed contracts lock in salaries based on projected (not realized) performance, introducing noise.

    \item \textbf{Non-WAR factors}: Teams may rationally pay for brand value, clubhouse presence, defensive positioning, or clutch performance not captured in WAR.

    \item \textbf{Risk preferences}: Risk-averse teams may discount volatile high-WAR players (explaining negative star premium).
\end{enumerate}

\subsection{Limitations and Future Work}

\begin{enumerate}
    \item \textbf{Data ends in 2016}: Post-2016 trends (luxury tax, tanking, analytics boom) are not captured.
    \item \textbf{No contract structure}: AAV, years, and option terms matter for true efficiency.
    \item \textbf{Revenue data limitations}: We use attendance as a proxy for market size; actual team revenues would improve the decision-theoretic analysis.
    \item \textbf{Static analysis}: We model team win values as fixed, but teams dynamically adjust strategies based on roster composition and competitive window.
\end{enumerate}

\section{Conclusion}

Using real MLB salary data and a Bayesian hierarchical model, we find substantial variation in team salary efficiency but no evidence of market-wide efficiency improvements from 2000--2016. The ``Moneyball'' revolution may have made all teams smarter, but it has not eliminated systematic pricing differences---small-market teams still achieve dramatically better value per dollar.

However, our decision-theoretic analysis reveals that apparent ``inefficiency'' is partly rational. Teams with high spending premiums tend to be large-market teams near playoff contention, where the marginal win value is genuinely higher. The correlation between implied and rational win values ($r = 0.50$, $p = 0.005$) suggests teams respond to incentives, even if imperfectly.

The key insight is that efficiency (WAR per dollar) and optimality (expected utility maximization) are distinct concepts. Oakland's ``efficiency'' may actually be suboptimal if they're leaving playoff probability on the table, while the Yankees' ``inefficiency'' may be rational given their revenue structure.

The absence of a star premium and the declining R$^2$ over time remain puzzling, suggesting that institutional constraints (service-time rules, guaranteed contracts) and behavioral factors continue to shape the market in ways not fully captured by rational models. The market for MLB talent is neither perfectly efficient nor completely irrational---it reflects a complex equilibrium of heterogeneous team objectives, institutional constraints, and imperfect information.

\appendix

\section{MCMC Diagnostics}

The Metropolis-within-Gibbs sampler ran for 4,000 iterations (1,000 warmup + 3,000 retained). Log posterior values showed adequate mixing:
\begin{itemize}
    \item Iteration 500: $-18,910$
    \item Iteration 2000: $-19,118$
    \item Iteration 4000: $-18,985$
\end{itemize}

\section{Data Matching}

Players matched between Lahman and FanGraphs on:
\begin{itemize}
    \item Lowercase first + last name
    \item Season
    \item Team (after standardizing abbreviations)
\end{itemize}

Match rate: approximately 50\% of eligible player-seasons.

\end{document}
