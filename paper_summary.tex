\documentclass[11pt]{article}

% ----------- Packages -----------
\usepackage[margin=1in]{geometry}
\usepackage{amsmath}
\usepackage{amssymb}
\usepackage{amsthm}
\usepackage{enumitem}
\usepackage{booktabs}
\usepackage{xcolor}
\usepackage{tcolorbox}
\usepackage{titlesec}

% ----------- Formatting -----------
\titlespacing*{\section}{0pt}{1.5ex}{1ex}
\titlespacing*{\subsection}{0pt}{1ex}{0.5ex}
\setlength{\parskip}{0.5em}
\setlength{\parindent}{0pt}

% ----------- Colors -----------
\definecolor{maroon}{RGB}{128,0,0}
\definecolor{newcolor}{RGB}{0,100,0}
\definecolor{knowncolor}{RGB}{100,100,100}

% ----------- Theorem Styles -----------
\newtheorem{theorem}{Theorem}
\newtheorem{proposition}{Proposition}
\theoremstyle{definition}
\newtheorem{definition}{Definition}

% ----------- Custom boxes -----------
\newtcolorbox{keyresult}{
  colback=maroon!5,
  colframe=maroon,
  fonttitle=\bfseries,
  title=Key Result
}

\newtcolorbox{newbox}{
  colback=newcolor!5,
  colframe=newcolor!70,
  fonttitle=\bfseries,
  title=New Contribution
}

% ----------- Commands -----------
\newcommand{\known}{\textcolor{knowncolor}{\textsc{[Known]}}}
\newcommand{\new}{\textcolor{newcolor}{\textsc{[New]}}}

% ----------- Document -----------
\begin{document}

\begin{center}
{\LARGE \textbf{Is Management Learning?}}\\[0.5em]
{\large Bayesian Decision Analysis of NFL Fourth Down and Two-Point Decisions}\\[1em]
{\textsc{Andrew Bai} \quad University of Chicago Booth}\\[0.3em]
{\today}
\end{center}

\vspace{-0.5em}
\rule{\textwidth}{0.4pt}

%===========================================================
\section{The Question}
%===========================================================

Two decades after Romer (2006) showed NFL teams are systematically too conservative on fourth down, \textbf{are coaches learning?}

\textbf{Three key sub-questions:}
\begin{itemize}[nosep]
  \item Are coaches making better decisions over time?
  \item Were optimal decisions \textit{knowable in real-time} (ex ante), or only with hindsight?
  \item How many games are teams losing from suboptimal decisions?
\end{itemize}

%===========================================================
\section{The Answer}
%===========================================================

\begin{keyresult}
\textbf{Coaches are not improving on fourth down.} Despite 20 years of analytics research:
\begin{itemize}[nosep]
  \item Ex ante optimal rate: \textbf{80.1\%} (flat trend, $-0.11$ pp/year, $p = 0.16$)
  \item \textbf{96.4\%} of optimal decisions were knowable in real-time
  \item Teams lose \textbf{0.27 expected wins/season} from fourth down errors
  \item League-wide WP loss \textbf{increasing}: 0.25 wins/team (2006--2015) $\to$ 0.35 (2022--2024)
\end{itemize}
\textbf{Contrast with two-point conversions:} Coaches \textit{are} improving (+1.0 pp/year, $p = 0.06$).

\textbf{Implication:} The analytics revolution changed \textit{behavior} (more aggression) but not \textit{accuracy}. Complexity matters for organizational learning.
\end{keyresult}

%===========================================================
\section{The Approach}
%===========================================================

\textbf{Bayesian decision-theoretic framework} that addresses limitations of prior work:

\begin{enumerate}[nosep]
  \item \textbf{Win probability optimization} (not expected points)---because variance matters
  \item \textbf{Full posterior uncertainty} propagated through to decision recommendations
  \item \textbf{Clock dynamics} via Bayesian clock consumption model
  \item \textbf{Expanding window analysis} to test real-time knowability
  \item \textbf{Hierarchical team effects} conditioning on offense/defense quality
\end{enumerate}

\textbf{Key insight:} The same framework applies to both fourth down and two-point decisions, but fourth down is \textit{harder}---more actions, more heterogeneity, higher stakes, clock dynamics.

%===========================================================
\section{Fourth Down: The Math}
%===========================================================

\textbf{Decision problem:} At each fourth down, coach chooses $a \in \{\texttt{go}, \texttt{punt}, \texttt{fg}\}$ to maximize win probability.

\textbf{State space:} $s = (\text{score\_diff}, \text{field\_pos}, \text{yards\_to\_go}, \text{time}, \text{timeouts})$

\textbf{Win probability for each action:}

\begin{align}
\text{WP}_{\texttt{go}}(s) &= p_{\text{convert}}(s) \cdot \text{WP}(s' | \text{convert}) + (1 - p_{\text{convert}}(s)) \cdot \text{WP}(s' | \text{fail}) \\[0.5em]
\text{WP}_{\texttt{punt}}(s) &= \text{WP}(s' | \text{punt}) \\[0.5em]
\text{WP}_{\texttt{fg}}(s) &= p_{\text{make}}(s) \cdot \text{WP}(s' | \text{make}) + (1 - p_{\text{make}}(s)) \cdot \text{WP}(s' | \text{miss})
\end{align}

\textbf{Optimal action:} $a^* = \arg\max_{a} \text{WP}_a(s)$

\textbf{Model components} \known:
\begin{itemize}[nosep]
  \item $p_{\text{convert}}$: Hierarchical logistic regression on yards to go
  \item $p_{\text{punt}}$: Normal model for net punt yards (mean $\approx$ 40 yds)
  \item $p_{\text{make}}$: Logistic regression on kick distance
  \item $\text{WP}$: Logistic model on score diff, time, field position, timeouts
\end{itemize}

\textbf{New: Bayesian clock consumption model} \new:
\begin{itemize}[nosep]
  \item Go-convert: $\mu = 200$s consumed (high variance: $\sigma = 144$s)
  \item Go-fail: $\mu = 135$s consumed (turnover on downs)
  \item Punt: $\mu = 178$s consumed
  \item FG-make: $\mu = 162$s consumed
  \item FG-miss: $\mu = 155$s consumed
\end{itemize}
Clock asymmetry is critical for late-game decisions.

%===========================================================
\section{Fourth Down: Learning Over Time}
%===========================================================

\subsection*{Ex Ante vs Ex Post Optimal Decisions}

\begin{newbox}
\textbf{Expanding window design:} For each test year $Y$, train models on 1999--$(Y-1)$, then evaluate year $Y$ decisions. This tests whether optimal decisions were \textit{knowable at the time}.
\end{newbox}

\begin{center}
\begin{tabular}{lcccc}
\toprule
\textbf{Year} & \textbf{N Plays} & \textbf{Ex Ante Optimal} & \textbf{Ex Post Optimal} & \textbf{Agreement} \\
\midrule
2006 & 3,829 & 81.3\% & 77.8\% & 93.5\% \\
2010 & 3,840 & 81.9\% & 78.5\% & 94.4\% \\
2015 & 3,824 & 78.8\% & 80.3\% & 96.6\% \\
2020 & 3,402 & 79.8\% & 79.1\% & 97.3\% \\
2024 & 3,827 & 78.3\% & 78.4\% & 97.8\% \\
\midrule
\textit{Overall} & 71,786 & \textbf{80.1\%} & 79.2\% & \textbf{96.4\%} \\
\bottomrule
\end{tabular}
\end{center}

\textbf{Key finding:} 96.4\% of optimal decisions were knowable in real-time. Hindsight explains almost nothing.

\subsection*{Trend Analysis}

\textbf{No improvement over time:}
\begin{itemize}[nosep]
  \item Linear trend: $-0.11$ pp/year ($p = 0.16$)---not statistically significant
  \item Early period (2006--2012): 80.8\% optimal
  \item Late period (2019--2024): 79.0\% optimal
\end{itemize}

\textbf{Coaches changed behavior but not accuracy:}
\begin{itemize}[nosep]
  \item Go-for-it rate increased dramatically (especially 2018+)
  \item But ``aggressive when should be conservative'' errors offset ``conservative when should be aggressive'' errors
  \item Net effect: no improvement in decision quality
\end{itemize}

%===========================================================
\section{Robustness: What About Fourth and Inches?}
%===========================================================

\textbf{Concern:} Are results driven by close calls where reasonable people disagree?

\begin{center}
\begin{tabular}{lccc}
\toprule
\textbf{Decision Clarity} & \textbf{N Plays} & \textbf{\% of Total} & \textbf{\% of Mistakes} \\
\midrule
Close call ($<$2\% margin) & 38,602 & 53.8\% & 84.0\% \\
Moderate (2--5\% margin) & 22,197 & 30.9\% & 12.9\% \\
Clear ($>$5\% margin) & 10,987 & 15.3\% & 3.0\% \\
\bottomrule
\end{tabular}
\end{center}

\textbf{Finding:} 84\% of ``mistakes'' are close calls where margin $<$ 2\%. Only 3\% of mistakes are clear errors. Most disagreements are \textit{defensible}.

\textbf{Fourth and one specifically:}
\begin{itemize}[nosep]
  \item 4th \& 1 from opponent's 35--45: Model says go, coaches often punt/kick
  \item These are the highest-leverage close calls
  \item Even here, margins are typically 1--3\% WP
\end{itemize}

%===========================================================
\section{Games Lost: Connecting to Romer (2006)}
%===========================================================

\begin{newbox}
\textbf{Expected wins lost by team and season} (expanding window, ex ante optimal):
\begin{center}
\begin{tabular}{lcccccc}
\toprule
\textbf{Team} & \textbf{'06} & \textbf{'10} & \textbf{'15} & \textbf{'20} & \textbf{'24} & \textbf{Avg} \\
\midrule
CLE (worst) & 0.27 & 0.44 & 0.27 & 0.34 & 0.76 & \textbf{0.34} \\
NYJ & 0.33 & 0.65 & 0.22 & 0.12 & 0.51 & 0.33 \\
TEN & 0.38 & 0.28 & 0.53 & 0.31 & 0.31 & 0.33 \\
\midrule
SF (best) & 0.27 & 0.23 & 0.19 & 0.28 & 0.37 & \textbf{0.22} \\
NO & 0.23 & 0.15 & 0.24 & 0.17 & 0.27 & 0.22 \\
DAL & 0.12 & 0.28 & 0.40 & 0.24 & 0.31 & 0.23 \\
\midrule
\textit{League Avg} & 0.25 & 0.28 & 0.24 & 0.27 & 0.37 & 0.27 \\
\bottomrule
\end{tabular}
\end{center}
\end{newbox}

\textbf{Comparison to Romer (2006):} Romer estimated teams left $\sim$0.4 expected wins/season on the table. Our estimate of 0.27 wins/season is \textit{slightly lower} but in the same ballpark. The difference:
\begin{itemize}[nosep]
  \item Romer: Expected points framework
  \item This paper: Win probability framework with uncertainty
  \item WP framework is more conservative (close calls have smaller margins)
\end{itemize}

\textbf{Troubling trend:} League-wide WP loss is \textit{increasing}:
\begin{itemize}[nosep]
  \item 2006--2015 average: 0.25 wins/team/season
  \item 2022--2024 average: 0.35 wins/team/season
  \item 2024: 11.9 total league wins lost (vs 7.9 in 2006)
\end{itemize}

The ``free wins'' Romer identified are \textit{still on the table}---and growing.

%===========================================================
\section{The Worst Decisions of All Time}
%===========================================================

\begin{center}
\small
\begin{tabular}{llllrl}
\toprule
\textbf{Year} & \textbf{Game} & \textbf{Situation} & \textbf{Did} & \textbf{Should} & \textbf{WP Cost} \\
\midrule
2021 & BUF @ TEN & 4th \& 1, opp 3, $-3$, 0:22 Q4 & Go & FG & 34.3\% \\
2016 & BAL @ JAX & 4th \& 2, own 49, $-1$, 2:14 Q4 & Go & Punt & 26.8\% \\
2020 & DAL @ SEA & 4th \& 3, own 47, $-1$, 2:37 Q4 & Go & Punt & 26.7\% \\
2012 & DAL @ CAR & 4th \& 1, own 40, $-2$, 2:11 Q4 & Go & Punt & 26.0\% \\
2020 & DEN @ NYJ & 4th \& 3, own 50, $-2$, 2:05 Q4 & Go & Punt & 24.8\% \\
\bottomrule
\end{tabular}
\end{center}
\normalsize

\textbf{Pattern:} The costliest mistakes are:
\begin{itemize}[nosep]
  \item Late-game situations with high leverage (under 3 minutes, Q4)
  \item ``Hero ball'' going for it when should punt to play defense
  \item Trailing by 1--3 points where field position matters
\end{itemize}

%===========================================================
\section{Two-Point Conversions: A Simpler Decision}
%===========================================================

\textbf{The math} (same framework, simpler case):

After a touchdown, choose $a \in \{\text{PAT}, \text{2pt}\}$:
\begin{align}
\text{WP}_{\text{PAT}} &= p_{\text{PAT}} \cdot \text{WP}(+7) + (1 - p_{\text{PAT}}) \cdot \text{WP}(+6) \\
\text{WP}_{\text{2pt}} &= p_{\text{2pt}} \cdot \text{WP}(+8) + (1 - p_{\text{2pt}}) \cdot \text{WP}(+6)
\end{align}

where $p_{\text{PAT}} \approx 94\%$ and $p_{\text{2pt}} \approx 48\%$.

\textbf{Why this is simpler than fourth down:}
\begin{itemize}[nosep]
  \item Only 2 actions (not 3)
  \item No field position dependence
  \item No clock dynamics (kickoff follows either way)
  \item Minimal team heterogeneity ($\tau \approx 0.03$ vs $\tau \approx 0.15$ for 4th down)
  \item Maximum WP margins $\sim$5\% (vs $>$30\% for fourth down)
\end{itemize}

\subsection*{Two-Point Results: Coaches ARE Learning}

\begin{center}
\begin{tabular}{lccccc}
\toprule
\textbf{Year} & \textbf{N} & \textbf{Ex Ante Optimal} & \textbf{Agreement} & \textbf{Actual 2pt\%} & \textbf{Optimal 2pt\%} \\
\midrule
2016 & 1,361 & 43.6\% & 61.6\% & 8.3\% & 59.4\% \\
2018 & 1,418 & 57.1\% & 60.9\% & 9.7\% & 42.3\% \\
2020 & 1,538 & 55.5\% & 71.6\% & 9.3\% & 47.7\% \\
2022 & 1,381 & 56.0\% & 74.2\% & 9.3\% & 46.7\% \\
2024 & 1,450 & 54.7\% & 87.0\% & 10.2\% & 50.1\% \\
\midrule
\textit{Average} & 1,407 & \textbf{54.2\%} & 70.4\% & 9.2\% & 47.8\% \\
\bottomrule
\end{tabular}
\end{center}

\textbf{Key findings:}
\begin{itemize}[nosep]
  \item Trend: $+1.0$ pp/year improvement ($p = 0.06$)---marginally significant
  \item Teams go for 2 only 9\% of the time, but optimal rate is $\sim$48\%
  \item Still \textit{way too conservative}, but improving
  \item Ex ante/ex post agreement improving as models stabilize
\end{itemize}

\textbf{Why are coaches learning here but not on fourth down?}
\begin{itemize}[nosep]
  \item Simpler decision space (binary, no field position)
  \item Lower stakes per decision (max 5\% WP swing)
  \item Clearer feedback signal (immediate outcome)
  \item Less room for ``coach knows best'' justification
\end{itemize}

%===========================================================
\section{Contributions Summary}
%===========================================================

\renewcommand{\arraystretch}{1.3}
\begin{center}
\begin{tabular}{p{7cm}c}
\toprule
\textbf{Contribution} & \textbf{Status} \\
\midrule
Fourth down is suboptimal (Romer 2006) & \known \\
Win probability $>$ expected points for decisions & \known \\
\midrule
Bayesian framework with full posterior uncertainty & \new \\
Expanding window (ex ante vs ex post) design & \new \\
96.4\% of optimal decisions knowable in real-time & \new \\
No improvement in fourth down decision quality & \new \\
Bayesian clock consumption model & \new \\
Team-level wins lost analysis & \new \\
Two-point comparison (learning \textit{does} occur) & \new \\
Complexity $\to$ organizational learning relationship & \new \\
\bottomrule
\end{tabular}
\end{center}

%===========================================================
\section{Implications}
%===========================================================

\textbf{For management research:}
\begin{itemize}[nosep]
  \item Information availability $\neq$ information use
  \item Complexity inhibits organizational learning, even with clear feedback
  \item ``Analytics revolution'' changed behavior but not accuracy
\end{itemize}

\textbf{For NFL teams:}
\begin{itemize}[nosep]
  \item 0.27 wins/season still on the table---essentially free
  \item Worse teams (CLE, NYJ) lose more from bad decisions
  \item The gap is \textit{growing}, not shrinking
\end{itemize}

\textbf{For empirical strategy:}
\begin{itemize}[nosep]
  \item High-frequency, observable decisions are ideal for studying organizational learning
  \item Bayesian framework enables principled uncertainty quantification
  \item Expanding window design separates ``knowable'' from ``hindsight''
\end{itemize}

%===========================================================
\section{The Down 8 vs Down 9 Paradox}
%===========================================================

\begin{newbox}
\textbf{A natural experiment in behavioral economics:} When should coaches go for 2?

\begin{center}
\begin{tabular}{lcccc}
\toprule
\textbf{Situation} & \textbf{N} & \textbf{Model: 2pt} & \textbf{Actual 2pt} & \textbf{Go When Should} \\
\midrule
Down 8 $\to$ Down 2 & 175 & 72\% & \textbf{81\%} & \textbf{84\%} \\
Down 9 $\to$ Down 3 & 105 & 72\% & 1\% & \textbf{0\%} \\
\midrule
Down 14 $\to$ Down 8 & 404 & 73\% & 9\% & 13\% \\
Down 15 $\to$ Down 9 & 87 & 77\% & 25\% & 25\% \\
Down 7 $\to$ Down 1 & 1,022 & 59\% & 4\% & 3\% \\
Tied $\to$ Up 6 & 2,330 & 46\% & 1\% & 1\% \\
\bottomrule
\end{tabular}
\end{center}
\end{newbox}

\textbf{The paradox:}
\begin{itemize}[nosep]
  \item \textbf{Down 8:} Going for 2 \textit{ties the game immediately}. Coaches do this 81\% of the time.
  \item \textbf{Down 9:} Going for 2 means a \textit{field goal can tie later}. Coaches do this 1\% of the time.
  \item Both situations have nearly identical optimal 2pt rates (72\%)!
\end{itemize}

\textbf{Behavioral explanation:}
\begin{itemize}[nosep]
  \item \textbf{Present bias:} Immediate payoff (tie NOW) vs deferred payoff (FG ties LATER)
  \item \textbf{Probability neglect:} Coaches underweight the probability of needing the FG
  \item \textbf{Salience:} ``Go for 2 to tie'' is a known heuristic; ``go for 2 so FG ties'' requires calculation
\end{itemize}

\textbf{Magnitude:} In 76 situations where the model said ``go for 2'' when down 9, coaches kicked the PAT \textit{every single time}. This is not a close call---it's a systematic blind spot.

%===========================================================
\section{Suggestions for Additional Checks}
%===========================================================

\textbf{Things to potentially verify or extend:}

\begin{enumerate}[nosep]
  \item \textbf{Coach-level analysis:} Do some coaches learn faster than others? Do ``analytics-friendly'' coaches (e.g., Pederson, Staley) actually make better decisions?

  \item \textbf{Playoff vs regular season:} Are coaches more conservative in high-stakes games? Does decision quality change?

  \item \textbf{Score differential interactions:} Are coaches worse when trailing (``desperate'' decisions) or leading (``playing it safe'')?

  \item \textbf{Weather/dome effects:} Does field goal accuracy uncertainty affect optimal decisions? (Current model uses league-average FG rates)

  \item \textbf{Fourth and short specifically:} Deep dive on 4th \& 1--2 decisions, which are the most controversial.

  \item \textbf{Pre/post analytics era:} Compare decision quality before vs after the ``Moneyball'' era ($\sim$2015).

  \item \textbf{Two-point game situations:} When down 2, 9, or 15 points, the math strongly favors going for 2. How often do coaches get these right?
\end{enumerate}

\vspace{1em}
\rule{\textwidth}{0.4pt}
\begin{center}
\textit{Feedback welcome.}
\end{center}

\end{document}
