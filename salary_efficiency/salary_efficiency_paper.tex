\documentclass[11pt]{article}
\usepackage[margin=1in]{geometry}
\usepackage{amsmath,amssymb,amsthm}
\usepackage{graphicx}
\usepackage{booktabs}
\usepackage{natbib}
\usepackage{hyperref}
\usepackage{float}

\title{MLB Salary Efficiency Analysis: \\A Bayesian Hierarchical Approach}
\author{}
\date{\today}

\begin{document}

\maketitle

\begin{abstract}
We analyze MLB salary efficiency using real contract data from the Lahman database (2000--2016) merged with FanGraphs WAR. Our analysis addresses four questions: (1) How has the market rate for wins evolved? (2) Are certain player types systematically mispriced? (3) Which teams achieve better salary efficiency? (4) Has the market become more efficient over time? Key findings from 8,198 player-seasons: the market rate rose from \$1.3M/WAR (2000) to \$2.8M/WAR (2016); pitchers command a 15\% premium over position players per WAR; star players ($>$5 WAR) are actually paid \emph{less} per WAR than average players, contradicting conventional wisdom; market-clearing teams (Oakland, Miami, Tampa Bay) pay 40--60\% less per WAR than large-market clubs; and contrary to the ``Moneyball'' narrative, market efficiency has \emph{not} improved over this period---R$^2$ of salary-WAR regressions actually declined slightly.
\end{abstract}

\section{Introduction}

Professional sports provide a natural laboratory for studying labor market efficiency. Unlike most markets, player productivity in baseball is precisely measured through statistics like Wins Above Replacement (WAR), and salaries are publicly observable. This paper asks four questions:

\begin{enumerate}
    \item What is the market rate for wins, and how has it evolved?
    \item Are certain positions or player types systematically mispriced?
    \item Which teams achieve better value in the salary market?
    \item Has market efficiency improved over time (``Does management learn?'')
\end{enumerate}

\subsection{Contribution}

We improve on naive salary-WAR regressions by using:
\begin{itemize}
    \item \textbf{Real salary data} from the Lahman database (not simulated)
    \item \textbf{Latent talent model} treating WAR as a noisy proxy for true value
    \item \textbf{Random walk year effects} for smooth market inflation
    \item \textbf{Piecewise WAR effects} to capture star premiums
    \item \textbf{Heteroskedastic residuals} scaling with player value
\end{itemize}

\subsection{Limitations}

Several caveats apply:
\begin{itemize}
    \item Lahman salary data ends in 2016; recent trends are not captured.
    \item We observe annual salary, not contract terms (years, guarantees, deferrals).
    \item ``Efficiency'' means salary per WAR, not team optimality---large-market teams may rationally pay premiums.
    \item WAR embeds modeling assumptions about replacement level.
\end{itemize}

\section{Data}

\subsection{Sources}

We merge two datasets:
\begin{itemize}
    \item \textbf{Lahman Salaries}: Real observed player salaries, 2000--2016
    \item \textbf{FanGraphs WAR}: Player productivity metrics via \texttt{pybaseball}
\end{itemize}

Matching on player name and team-year yields 8,198 player-seasons with WAR $\geq 0.1$.

\subsection{Sample Statistics}

\begin{table}[H]
\centering
\caption{Sample Statistics (Real Salary Data)}
\label{tab:sample}
\begin{tabular}{lr}
\toprule
Metric & Value \\
\midrule
Total player-seasons & 8,198 \\
Unique teams & 30 \\
Seasons & 2000--2016 \\
\midrule
Salary distribution: & \\
\quad Min & \$165,574 \\
\quad Median & \$1,600,000 \\
\quad Mean & \$3,724,070 \\
\quad Max & \$33,000,000 \\
\midrule
WAR distribution: & \\
\quad Mean & 1.96 \\
\quad Median & 1.40 \\
\quad Range & [0.1, 12.7] \\
\bottomrule
\end{tabular}
\end{table}

\section{Descriptive Analysis}

\subsection{Market Rate for Wins}

The market rate (total salary / total WAR) has more than doubled over our sample period:

\begin{table}[H]
\centering
\caption{Market Rate for Wins Over Time}
\label{tab:marketrate}
\begin{tabular}{lrrr}
\toprule
Year & \$/WAR & Avg Salary & N \\
\midrule
2000 & \$1.26M & \$2.47M & 488 \\
2004 & \$1.54M & \$3.11M & 481 \\
2008 & \$1.88M & \$3.67M & 480 \\
2012 & \$2.15M & \$4.02M & 489 \\
2016 & \$2.76M & \$5.11M & 499 \\
\bottomrule
\end{tabular}
\end{table}

The \$/WAR grew 118\% from 2000 to 2016, reflecting both revenue growth and competitive bidding for talent.

\subsection{Position Pricing: Pitchers vs.\ Batters}

\begin{table}[H]
\centering
\caption{Position Pricing}
\label{tab:position}
\begin{tabular}{lrrr}
\toprule
Position & \$/WAR & Avg Salary & N \\
\midrule
Pitchers & \$2.08M & \$3.13M & 3,595 \\
Batters & \$1.81M & \$4.19M & 4,603 \\
\bottomrule
\end{tabular}
\end{table}

\textbf{Finding}: Pitchers are paid 15\% \emph{more} per WAR than position players. This may reflect:
\begin{itemize}
    \item Higher injury risk requiring risk premium
    \item Greater volatility in pitcher performance
    \item Scarcity of quality starting pitching
\end{itemize}

\subsection{Star Premium (or Lack Thereof)}

Conventional wisdom holds that star players command a premium per win due to scarcity. Our data tell a different story:

\begin{table}[H]
\centering
\caption{Star Premium: \$/WAR by Performance Tier}
\label{tab:star}
\begin{tabular}{lrrr}
\toprule
WAR Bucket & \$/WAR & Avg Salary & N \\
\midrule
0--1 WAR & \$4.67M & \$2.39M & 3,245 \\
1--2 WAR & \$2.21M & \$3.33M & 1,901 \\
2--3 WAR & \$1.78M & \$4.47M & 1,209 \\
3--4 WAR & \$1.46M & \$5.09M & 815 \\
4--5 WAR & \$1.33M & \$6.01M & 488 \\
5+ WAR & \$1.15M & \$7.34M & 540 \\
\bottomrule
\end{tabular}
\end{table}

\textbf{Key finding}: The \$/WAR \emph{decreases} monotonically with performance. Stars (5+ WAR) are paid only \$1.15M per WAR compared to \$4.67M for marginal players (0--1 WAR). This is a 4:1 ratio in the \emph{opposite} direction of conventional wisdom.

\textbf{Explanation}: This reflects \emph{selection effects}. Low-WAR players in our sample are overpaid veterans on guaranteed contracts whose performance declined, pre-arbitration minimums for players who barely contributed, or injury cases. High-WAR players include cost-controlled pre-arb and arbitration players (like Mike Trout earning \$500K while producing 9 WAR).

\subsection{Does Management Learn? Market Efficiency Over Time}

We test whether the salary market has become more efficient by examining how well salary predicts WAR (higher R$^2$ = more efficient pricing):

\begin{table}[H]
\centering
\caption{Market Efficiency by Era}
\label{tab:efficiency}
\begin{tabular}{lrrr}
\toprule
Era & R$^2$ & Residual Var & N \\
\midrule
2000--2004 & 0.127 & 1.53 & 2,401 \\
2005--2008 & 0.065 & 1.62 & 1,924 \\
2009--2012 & 0.076 & 1.53 & 1,901 \\
2013--2016 & 0.073 & 1.58 & 1,972 \\
\bottomrule
\end{tabular}
\end{table}

\textbf{Key finding}: Market efficiency has \emph{not} improved. The R$^2$ of salary-WAR regressions actually \emph{declined} from 0.127 in 2000--2004 to 0.073 in 2013--2016.

Formal trend test:
\begin{itemize}
    \item R$^2$ trend: $-0.0038$/year ($p = 0.008$)
    \item RMSE trend: $+0.0008$/year ($p = 0.61$)
\end{itemize}

The negative R$^2$ trend is statistically significant, contradicting the ``Moneyball'' narrative that analytics have made the market more efficient. Possible explanations:
\begin{itemize}
    \item Guaranteed contracts lock in salaries regardless of performance
    \item Arbitration/pre-arb system decouples salary from current productivity
    \item Teams may value projected future performance, not current WAR
    \item Risk preferences and injury history affect pricing
\end{itemize}

\section{Hierarchical Model for Team Efficiency}

\subsection{Model Specification}

We model log salary as a function of latent talent with team and year effects:

\textbf{Latent talent (measurement error model):}
\begin{equation}
    \text{WAR}_i \mid \theta_i \sim N(\theta_i, \sigma_{\text{WAR}}^2)
\end{equation}

\textbf{Salary model with piecewise WAR effect:}
\begin{equation}
    \log(\text{salary}_i) = \alpha + \beta_1 \theta_i + \beta_2 (\theta_i - 3)_+ + \gamma_{\text{team}}[t_i] + \gamma_{\text{year}}[y_i] + \epsilon_i
\end{equation}

\textbf{Random walk year effects:}
\begin{equation}
    \gamma_{\text{year}}[t] \mid \gamma_{\text{year}}[t-1] \sim N(\gamma_{\text{year}}[t-1], \tau^2)
\end{equation}

\textbf{Heteroskedastic residuals:}
\begin{equation}
    \log \sigma_i = a + b \cdot \theta_i
\end{equation}

\subsection{Results: Fixed Effects}

\begin{table}[H]
\centering
\caption{Posterior Estimates for Fixed Effects}
\label{tab:fixed}
\begin{tabular}{lrrr}
\toprule
Parameter & Post.\ Mean & Post.\ SD & Interpretation \\
\midrule
$\alpha$ (intercept) & 13.73 & 0.14 & Baseline log salary \\
$\beta_1$ (base WAR) & 0.28 & 0.02 & 1.32$\times$ per WAR (below 3) \\
$\beta_2$ (star effect) & $-$0.12 & 0.03 & 1.17$\times$ per WAR (above 3) \\
\bottomrule
\end{tabular}
\end{table}

The negative $\beta_2$ confirms the descriptive finding: salary growth per WAR \emph{slows} for elite players. Each additional WAR multiplies salary by 1.32$\times$ for players below 3 WAR, but only 1.17$\times$ above that threshold.

\subsection{Results: Team Efficiency Rankings}

Team efficiency = $\exp(-\gamma_{\text{team}})$: teams with negative $\gamma$ pay less for equivalent talent.

\begin{table}[H]
\centering
\caption{Team Efficiency Rankings (Top/Bottom 5)}
\label{tab:teams}
\begin{tabular}{lrrrr}
\toprule
Team & $\gamma$ Mean & Efficiency & 95\% CI & P(Above Avg) \\
\midrule
\multicolumn{5}{l}{\textit{Most Efficient (pay less per WAR):}} \\
MIA & $-$0.45 & 1.58 & [1.33, 1.88] & 100\% \\
OAK & $-$0.41 & 1.51 & [1.26, 1.80] & 100\% \\
PIT & $-$0.37 & 1.45 & [1.21, 1.73] & 100\% \\
TBR & $-$0.36 & 1.44 & [1.19, 1.72] & 100\% \\
SDP & $-$0.27 & 1.31 & [1.12, 1.57] & 99.8\% \\
\midrule
\multicolumn{5}{l}{\textit{Least Efficient (pay more per WAR):}} \\
CHC & 0.29 & 0.75 & [0.63, 0.88] & 0.0\% \\
NYM & 0.31 & 0.74 & [0.62, 0.87] & 0.0\% \\
LAD & 0.38 & 0.68 & [0.57, 0.82] & 0.0\% \\
BOS & 0.46 & 0.63 & [0.53, 0.75] & 0.0\% \\
NYY & 0.75 & 0.48 & [0.40, 0.56] & 0.0\% \\
\bottomrule
\end{tabular}
\end{table}

The efficiency gap is substantial: Oakland obtains roughly 3$\times$ more WAR per dollar than the Yankees. This reflects both payroll constraints and organizational philosophy.

\textbf{Caveat}: ``Inefficiency'' may be rational for large-market teams if their marginal win value is higher (playoff odds, revenue, brand value). The Yankees paying premium prices may still be optimal given their revenue structure.

\subsection{Results: Year Effects (Market Inflation)}

\begin{table}[H]
\centering
\caption{Year Effects (Random Walk)}
\label{tab:years}
\begin{tabular}{lrr}
\toprule
Year & $\gamma_{\text{year}}$ Mean & 95\% CI \\
\midrule
2000 & $-$0.23 & [$-$0.43, 0.10] \\
2004 & $-$0.07 & [$-$0.27, 0.26] \\
2008 & 0.07 & [$-$0.12, 0.41] \\
2012 & 0.22 & [0.02, 0.56] \\
2016 & 0.40 & [0.20, 0.73] \\
\bottomrule
\end{tabular}
\end{table}

Controlling for WAR and team, salaries increased by $\exp(0.63) \approx 1.9\times$ from 2000 to 2016, or about 4\% annually.

\section{Discussion}

\subsection{Summary of Findings}

\begin{enumerate}
    \item \textbf{Market rate doubled}: \$/WAR grew from \$1.3M (2000) to \$2.8M (2016).

    \item \textbf{Pitchers command premium}: Pitchers are paid 15\% more per WAR than position players, likely reflecting injury risk and volatility.

    \item \textbf{No star premium}: Contrary to conventional wisdom, stars are paid \emph{less} per WAR than average players, due to service-time rules and selection effects.

    \item \textbf{Large team effects}: Market-clearing teams (OAK, TBR, MIA) pay 40--60\% less per WAR than large-market clubs (NYY, BOS, LAD).

    \item \textbf{No efficiency improvement}: Market efficiency has not improved over time; R$^2$ of salary-WAR regressions actually declined.
\end{enumerate}

\subsection{Why Hasn't the Market Become More Efficient?}

The ``Moneyball'' narrative suggests that as analytics spread, inefficiencies should be arbitraged away. Our data show the opposite. Possible explanations:

\begin{enumerate}
    \item \textbf{Institutional constraints}: The service-time system (pre-arb, arbitration, free agency) prevents salaries from clearing at market rates. Cost-controlled players produce surplus value regardless of efficiency.

    \item \textbf{Contract structure}: Multi-year guaranteed contracts lock in salaries based on projected (not realized) performance, introducing noise.

    \item \textbf{Non-WAR factors}: Teams may rationally pay for brand value, clubhouse presence, defensive positioning, or clutch performance not captured in WAR.

    \item \textbf{Risk preferences}: Risk-averse teams may discount volatile high-WAR players (explaining negative star premium).
\end{enumerate}

\subsection{Limitations and Future Work}

\begin{enumerate}
    \item \textbf{Data ends in 2016}: Post-2016 trends (luxury tax, tanking, analytics boom) are not captured.
    \item \textbf{No contract structure}: AAV, years, and option terms matter.
    \item \textbf{No decision-theoretic framing}: Future work should model team utility (playoff probability, revenue) to evaluate optimality, not just efficiency.
\end{enumerate}

\section{Conclusion}

Using real MLB salary data and a Bayesian hierarchical model, we find substantial variation in team salary efficiency but no evidence of market-wide efficiency improvements from 2000--2016. The ``Moneyball'' revolution may have made all teams smarter, but it has not eliminated systematic pricing differences---small-market teams still achieve dramatically better value per dollar.

The absence of a star premium and the persistence of team effects suggest that institutional constraints (service-time rules, guaranteed contracts) dominate any efficiency gains from analytics. The market for MLB talent remains far from efficient.

\appendix

\section{MCMC Diagnostics}

The Metropolis-within-Gibbs sampler ran for 4,000 iterations (1,000 warmup + 3,000 retained). Log posterior values showed adequate mixing:
\begin{itemize}
    \item Iteration 500: $-18,910$
    \item Iteration 2000: $-19,118$
    \item Iteration 4000: $-18,985$
\end{itemize}

\section{Data Matching}

Players matched between Lahman and FanGraphs on:
\begin{itemize}
    \item Lowercase first + last name
    \item Season
    \item Team (after standardizing abbreviations)
\end{itemize}

Match rate: approximately 50\% of eligible player-seasons.

\end{document}
