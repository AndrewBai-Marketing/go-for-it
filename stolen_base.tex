\documentclass[11pt]{article}
\usepackage[margin=1in]{geometry}
\usepackage{amsmath,amssymb,amsthm}
\usepackage{graphicx}
\usepackage{booktabs}
\usepackage{natbib}
\usepackage{hyperref}
\usepackage{float}

\title{Stolen Base Decision Analysis: \\A Bayesian Decision-Theoretic Framework}
\author{}
\date{\today}

\begin{document}

\maketitle

\begin{abstract}
We develop a Bayesian decision-theoretic framework to evaluate stolen base decisions in Major League Baseball. Using pitch-level Statcast data from 2023--2024, we identify 91,755 steal opportunities and 767 actual attempts. We estimate a Bayesian logistic regression model for steal success probability conditional on game state using MCMC, then compare observed decisions against the break-even threshold derived from run expectancy tables. Our analysis finds that teams succeed at a 74.7\% rate, which exceeds the 72.2\% break-even threshold by 2.5 percentage points. Approximately 65.3\% of steal attempts have posterior mean success probabilities above break-even. These results suggest that MLB teams are appropriately selective when choosing to attempt stolen bases, correctly identifying high-probability situations.
\end{abstract}

\section{Introduction}

The stolen base decision in baseball presents a classic risk-reward tradeoff. A successful steal advances the runner, increasing expected runs. A failed attempt results in an out, decreasing expected runs and potentially ending a rally. The optimal decision depends on the probability of success relative to a ``break-even'' threshold determined by the run expectancy impact of success versus failure.

This paper applies Bayesian decision theory to evaluate whether MLB teams make optimal stolen base decisions. We address three questions:
\begin{enumerate}
    \item What is the break-even success probability for steal attempts by game state?
    \item Do teams' observed success rates exceed this threshold?
    \item Are teams appropriately selective in choosing when to attempt steals?
\end{enumerate}

\section{Data}

\subsection{Data Source}

We use pitch-level Statcast data obtained via the \texttt{pybaseball} package, covering the 2023 and 2024 MLB seasons. The raw dataset contains 1,437,933 pitches.

\subsection{Identifying Steal Opportunities}

A \textit{steal opportunity} is defined as a plate appearance where:
\begin{itemize}
    \item A runner occupies first or second base
    \item The next base is unoccupied (i.e., second base open for steals of second, third base open for steals of third)
    \item There are fewer than two outs
\end{itemize}

We aggregate pitch-level data to the plate appearance level, identifying whether a steal was attempted and whether it succeeded. Steal events are identified from the play description field using pattern matching for phrases such as ``steals 2nd,'' ``steals 3rd,'' ``caught stealing 2nd,'' etc.

\subsection{Sample Statistics}

Table~\ref{tab:sample} summarizes the dataset.

\begin{table}[H]
\centering
\caption{Sample Statistics}
\label{tab:sample}
\begin{tabular}{lr}
\toprule
Metric & Value \\
\midrule
Total steal opportunities & 91,755 \\
Steal attempts & 767 \\
Attempt rate & 0.84\% \\
Successful steals & 573 \\
Success rate (conditional on attempt) & 74.7\% \\
\midrule
By season: & \\
\quad 2023 opportunities & 46,619 \\
\quad 2023 attempts & 385 \\
\quad 2024 opportunities & 45,136 \\
\quad 2024 attempts & 382 \\
\bottomrule
\end{tabular}
\end{table}

\section{Model}

\subsection{Success Probability Model}

Let $Y_i \in \{0, 1\}$ denote the outcome of steal attempt $i$, where $Y_i = 1$ indicates success. We model the success probability as:
\begin{equation}
    P(Y_i = 1 \mid \mathbf{x}_i) = \sigma(\mathbf{x}_i^\top \boldsymbol{\beta})
\end{equation}
where $\sigma(z) = 1/(1 + e^{-z})$ is the logistic function and $\mathbf{x}_i$ is a vector of covariates.

\subsubsection{Covariates}

We include the following game-state covariates:
\begin{itemize}
    \item \textbf{Outs}: Number of outs (0 or 1), entered as a continuous variable
    \item \textbf{Inning}: Normalized as $(\text{inning} - 5) / 4$ to center around mid-game
    \item \textbf{Score differential}: Batting team's lead (positive) or deficit (negative)
    \item \textbf{Stealing third}: Indicator for whether the runner is attempting to steal third base (vs.\ second)
    \item \textbf{Left-handed pitcher}: Indicator for LHP on the mound
\end{itemize}

The model specification is:
\begin{equation}
    \text{logit}(p_i) = \beta_0 + \beta_1 \cdot \text{outs}_i + \beta_2 \cdot \text{inning}_i + \beta_3 \cdot \text{score\_diff}_i + \beta_4 \cdot \mathbf{1}[\text{steal 3B}]_i + \beta_5 \cdot \mathbf{1}[\text{LHP}]_i
\end{equation}

\subsubsection{Prior Specification}

We adopt weakly informative priors for all coefficients:
\begin{equation}
    \beta_j \stackrel{\text{iid}}{\sim} \mathcal{N}(0, 2.5^2), \quad j = 0, \ldots, p
\end{equation}
This prior centers coefficients at zero (no effect) with a standard deviation of 2.5, which on the log-odds scale allows for effects ranging from approximately $-5$ to $+5$ with high probability. This is weakly informative: it regularizes extreme estimates while remaining dominated by the likelihood for moderate sample sizes.

\subsubsection{Posterior Inference}

We sample from the posterior distribution using Metropolis-Hastings MCMC with adaptive proposal tuning:
\begin{equation}
    p(\boldsymbol{\beta} \mid \mathbf{y}) \propto \underbrace{\prod_{i=1}^{n} p_i^{y_i} (1 - p_i)^{1 - y_i}}_{\text{likelihood}} \times \underbrace{\prod_{j=0}^{p} \phi(\beta_j; 0, 2.5)}_{\text{prior}}
\end{equation}
where $p_i = \sigma(\mathbf{x}_i^\top \boldsymbol{\beta})$ and $\phi(\cdot; \mu, \sigma)$ denotes the normal density.

We run 1,000 warmup iterations (discarded) followed by 5,000 posterior samples. During warmup, proposal standard deviations are adapted to achieve approximately 44\% acceptance rates. Convergence is assessed via effective sample size calculations based on autocorrelation.

\subsection{Parameter Estimates}

Table~\ref{tab:model} reports the estimated coefficients.

\begin{table}[H]
\centering
\caption{Posterior Estimates for Success Probability Model}
\label{tab:model}
\begin{tabular}{lrrrl}
\toprule
Parameter & Post.\ Mean & Post.\ SD & 95\% Credible Interval & \\
\midrule
Intercept & 1.31 & 0.15 & $[1.02, 1.60]$ & * \\
Outs & $-0.42$ & 0.18 & $[-0.75, -0.08]$ & * \\
Inning & 0.36 & 0.13 & $[0.09, 0.61]$ & * \\
Score differential & $-0.28$ & 0.23 & $[-0.72, 0.17]$ & \\
Stealing 3rd & 0.99 & 0.34 & $[0.34, 1.69]$ & * \\
Left-handed pitcher & 0.28 & 0.23 & $[-0.16, 0.73]$ & \\
\bottomrule
\multicolumn{5}{l}{\footnotesize * 95\% credible interval excludes zero}
\end{tabular}
\end{table}

\subsubsection{Interpretation}

\begin{itemize}
    \item \textbf{Baseline success rate}: The intercept of 1.31 corresponds to a baseline success probability of $\sigma(1.31) = 78.8\%$.
    \item \textbf{Outs effect}: Each additional out decreases the log-odds by 0.43, corresponding to approximately 8 percentage points lower success probability. This may reflect that with one out, defenses are more willing to concede the base.
    \item \textbf{Inning effect}: Later innings are associated with higher success. A one-standard-deviation increase in inning (about 4 innings) increases success probability by approximately 7 percentage points.
    \item \textbf{Stealing third}: Attempts to steal third have significantly higher success rates (+0.94 log-odds, about 15 pp). This likely reflects extreme selection---teams only attempt steal of third in very favorable situations.
    \item \textbf{Pitcher handedness}: Left-handed pitchers do not significantly affect success probability in our sample.
\end{itemize}

\section{Break-Even Analysis}

\subsection{Run Expectancy Framework}

The break-even success probability $\pi^*$ is derived from run expectancy (RE24) tables. For a steal attempt to have positive expected value:
\begin{equation}
    \pi \cdot \text{RE}_{\text{success}} + (1 - \pi) \cdot \text{RE}_{\text{fail}} > \text{RE}_{\text{current}}
\end{equation}

Solving for the break-even:
\begin{equation}
    \pi^* = \frac{\text{RE}_{\text{current}} - \text{RE}_{\text{fail}}}{\text{RE}_{\text{success}} - \text{RE}_{\text{fail}}}
\end{equation}

\subsection{Break-Even by Situation}

Using standard run expectancy values, the break-even thresholds are:

\begin{table}[H]
\centering
\caption{Break-Even Success Probability by Situation}
\label{tab:breakeven}
\begin{tabular}{llc}
\toprule
Steal & Outs & Break-Even $\pi^*$ \\
\midrule
2nd base & 0 & 71.5\% \\
2nd base & 1 & 72.6\% \\
3rd base & 0 & 75.0\% \\
3rd base & 1 & 72.0\% \\
\bottomrule
\end{tabular}
\end{table}

The break-even is highest for stealing third with no outs (75\%) because the marginal value of advancing from second to third is relatively small compared to the cost of making an out with no outs.

\section{Results}

\subsection{Decision Quality}

We evaluate steal decisions by comparing predicted success probabilities to break-even thresholds.

\begin{table}[H]
\centering
\caption{Decision Analysis Summary}
\label{tab:decisions}
\begin{tabular}{lr}
\toprule
Metric & Value \\
\midrule
Steal attempts & 767 \\
Attempts above break-even & 501 (65.3\%) \\
Observed success rate & 74.7\% \\
Average break-even (for attempts) & 72.2\% \\
Success margin & +2.5 pp \\
\bottomrule
\end{tabular}
\end{table}

\subsection{Year-by-Year Analysis}

\begin{table}[H]
\centering
\caption{Steal Metrics by Season}
\label{tab:years}
\begin{tabular}{lccccc}
\toprule
Season & Opportunities & Attempts & Attempt Rate & Success Rate & Margin \\
\midrule
2023 & 46,619 & 385 & 0.83\% & 75.3\% & +3.1 pp \\
2024 & 45,136 & 382 & 0.85\% & 74.1\% & +1.9 pp \\
\bottomrule
\end{tabular}
\end{table}

\subsection{Interpretation}

Our key finding is that \textbf{teams appear to be appropriately selective} when choosing to attempt stolen bases:

\begin{enumerate}
    \item \textbf{Success rate exceeds break-even}: The observed 74.7\% success rate is 2.5 percentage points above the average break-even threshold of 72.2\%. This means steal attempts are, on average, positive expected value plays.

    \item \textbf{Majority of attempts are above break-even}: 65.3\% of steal attempts have posterior mean success probabilities exceeding the situation-specific break-even. The remaining 34.7\% may reflect (a) posterior uncertainty, (b) unobserved factors like runner speed or catcher arm strength, or (c) genuinely suboptimal decisions.

    \item \textbf{Selection effect}: The high success rate is \textit{expected} precisely because teams are selective. They only attempt steals when conditions are favorable---good jump, slow pitcher delivery, weak catcher arm, fast runner, etc. The 74.7\% success rate is not evidence that teams should steal more; rather, it reflects that teams correctly identify high-probability situations.
\end{enumerate}

\section{Discussion}

\subsection{Selection Bias}

A fundamental challenge in this analysis is selection bias. We observe success rates conditional on the decision to attempt a steal. Teams presumably only attempt steals when they believe the probability of success is high. Therefore, the observed 74.7\% success rate cannot be extrapolated to suggest that ``teams should steal more often.''

If teams attempted to steal in a random sample of opportunities, the success rate would likely be much lower. The appropriate interpretation is that teams are doing a good job of \textit{identifying when to steal}, not that they are stealing too infrequently.

\subsection{Unobserved Factors}

Our model captures game-state variables (outs, inning, score, base being stolen) but does not include:
\begin{itemize}
    \item Runner speed and base-stealing ability
    \item Catcher arm strength and pop time
    \item Pitcher delivery time and pickoff frequency
    \item Count and pitch type
    \item Element of surprise
\end{itemize}

These factors are presumably incorporated into teams' decisions, which explains why their realized success rate (74.7\%) exceeds what our simple model would predict for many attempts.

\subsection{Comparison to Fourth Down Analysis}

Unlike NFL fourth-down decisions---where teams have historically been too conservative---stolen base decisions appear to be approximately optimal. Several factors may explain this difference:

\begin{enumerate}
    \item \textbf{Continuous feedback}: Base stealers receive immediate feedback (safe/out), allowing rapid learning.
    \item \textbf{Individual skill}: Stolen bases depend heavily on individual runner ability, which is well-measured.
    \item \textbf{Clear decision criteria}: The break-even analysis for steals is straightforward and well-understood.
    \item \textbf{Smaller stakes}: A caught stealing, while costly, is not as psychologically salient as a failed fourth-down conversion.
\end{enumerate}

\section{Conclusion}

Using a Bayesian decision-theoretic framework, we find that MLB teams make approximately optimal stolen base decisions. The observed success rate of 74.7\% exceeds the break-even threshold of 72.2\%, and 65.3\% of attempts are predicted to be above break-even. Unlike NFL fourth-down decisions, where teams are demonstrably too conservative, stolen base decisions appear to reflect appropriate risk assessment.

The key limitation of this analysis is selection bias: we cannot observe counterfactual success rates for non-attempts. Future work could incorporate player-level covariates (sprint speed, catcher pop time) to better predict success probabilities and identify potential missed opportunities.

\appendix

\section{MCMC Diagnostics}

\subsection{Convergence}

We assess MCMC convergence using effective sample size (ESS) computed from autocorrelation. For all parameters, we obtain ESS $> 500$, indicating adequate mixing. The chains were initialized at the observed log-odds success rate for the intercept and zero for other coefficients.

\subsection{Posterior Predictive Check}

The posterior mean predicted success rate for attempts matches the observed rate (both approximately 75\%), indicating adequate model calibration. The primary value of the Bayesian approach is uncertainty quantification: the 95\% credible intervals properly reflect parameter uncertainty, which propagates to predictions.

\section{Data Processing Details}

Steal events were identified from the \texttt{des} (description) field in Statcast data using the following patterns:
\begin{itemize}
    \item Success: ``steals (2nd$|$3rd$|$home)''
    \item Failure: ``caught stealing (2nd$|$3rd$|$home)''
\end{itemize}

Plate appearances were aggregated from pitch-level data, with steal outcomes attached to the plate appearance in which they occurred.

\end{document}
